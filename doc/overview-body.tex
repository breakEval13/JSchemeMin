\chapter*{Scheme简介}

本文给出\rsevenrs 的小语言的概述.
此概述的目的为说明理解\rsevenrs{}报告所需的基本概念, 该报告被组织为参考手册.
因此, 本概述不是语言的完整介绍,也不在任何方面精确或在任何意义下规范.

\vest 跟随Algor,Scheme是一种静态作用域的语言。一个变量的每次使用关联一个词法决定的绑定。

\vest Scheme不同于显式类型. 类型与对象(又称值)而不是变量关联。(有些作者称隐式类型为类型、弱类型或动态类型。)隐式类型语言包括Python,Ruby,Smalltalk和其它Lisp方言。显式类型语言(有时称为强类型或静态类型语言)包括Algor 60,C,Java,C \#,Haskell,和ML.

\vest 所有在Scheme计算过程中创建的对象,包括过程和延续,有无限的层次。没有任何Scheme对象被销毀。Scheme实现(通常)不用光存储空间的原因是它们被允许回收可以证明不可能影响任何未来计算的对象占用的存储空间。其他语言中,大多数对象无限制的语言包括C\#,Java,Haskell,大多数的Lisp方言,ML,Python,Ruby,和Smalltalk。

\vest Scheme的实现必须正确地尾递归。这容许在常数空间中执行迭代计算,即使迭代计算的语法描述为递归过程。因此,用正确的尾部递归实现,可以用自调用机制来表示迭代,因此,特殊的迭代构造只是有用的句法糖。

\vest Scheme是第一种过程作为公民对象的语言之一支持程序的一种对象在自己的权利。程序可以动态创建,存储在数据结构中,作为返回值等等。有这性能的其他语言包括Common Lisp,Haskell,ML,Ruby,和Smalltalk。

\vest Scheme的一个标志性特性为继续,它在大多数语言只在背后工作, 而在这里有``头等''地位。头等继续对实现广泛的高级控制结构有用,包括非本地退出、回溯和协程。

在Scheme,参数在过程取得控制前求值,不管过程是否需要它们的值。 C、 C\#、 Common Lisp、 Python、Ruby和Smalltalk亦然。这与Haskell的惰性求值或Algol 60的按名调用语义不同,它们在过程需要时才求值参数表达式。

Scheme的算术模型提供丰富的数值类型和操作。进一步它区分\textit{精确}和\textit{非精确数}:实际上一个精确数精确对应于一个数,而非精确数为涉及舍入或其它近似的数。

\chapter{基本类型}

Scheme处理\textit{对象},也称为值\textit{值}。Scheme对象组织为称为\textit{类型}的值集合。本章给出Scheme语言重要基本类型的概述。

\begin{note}
  因为Scheme为隐式类型的,Scheme语景下 \textit{类型} 一词使用与其它语言特别是显式类型语言语景下不同。
\end{note}

\paragraph{数}

Scheme支持广泛的数值类型,包括任意精度整数、有理数、复数和多种非精确数。

\paragraph{布尔值}

布尔值即直值,可以是真或假。在Scheme, ``假''对象记为 \schfalse{},``真''对象记为 \schtrue{}。但在大多数要求真值的场合,非假的对象都被当作真。

\paragraph{序偶和列表}

序偶为有两个分量的数据结构。最常用于表示(单向链)列表,其中首分量(``car'')表示列表首个元素,次分量(``car'')表示余下的列表。Scheme 也有独特的空列表,用于列表中最后一个序偶的cdr。

\paragraph{符号}

符号是表示一个字符串(它的\textit{名字})的对象。与字符串不同,两个名字相同的符号不可区分。符号有很多应用,例如可用来模仿其它语言的枚举。

在 \rsevenrs,不同于 \rfivers,符号和标识符区分大小写。

\paragraph{字符}

Scheme字符大致对应于文本字符。更准确地,它同构于Unicode标准中\textit{标量值}的子集,可能有与实现有关的扩展。

\paragraph{字符串}

字符串为字符的定长序列,从而表示Unicode文本。

\paragraph{向量}

向量和列表同为表示任意对象的有限序列的的线性数据结构。列表的元素通过遍历序偶链来存取,而向量的元素由整数下标存取。因此,向量比列表更适合随机访问。

\paragraph{位向量}

位向量类似于向量,除了它的内容为\textit{字节}, 即从0到255的精确整数。

\paragraph{过程}

过程在Scheme为值。

\paragraph{记录}

记录为结构化的值,是零个或更多个域\textit{域}的聚合,每一个存放单个位置。记录组织为\textit{记录类型}。一个记录类型可定义一个谓词、构造器、域访问器、域修改器。

\paragraph{端口}

端口表示输入输出装置。对于Scheme,输入端口是一个可通过命令提供数据的Scheme对象,而输出端口是一个可接受数据的Scheme对象.

\chapter{表达式}

Scheme代码大部分主要元素为\textit{表达式}。表达式可被\textit{求值}, 提供 \textit{值} (实际上任意多个值)。最基本的表达式为字面值表达式:

\begin{scheme}
\schtrue{} \ev \schtrue
23 \ev 23%
\end{scheme}

这记号指 \schtrue{} 求值为\schtrue{},即``真''的值,而{\cf 23} 求值为表示23的数。

复合表达式由括号和子表达式组成。第一个子表达式确定操作; 其余子表达式为操作数:
%
\begin{scheme}
(+ 23 42) \ev 65
(+ 14 (* 23 42)) \ev 980%
\end{scheme}
%
在第一个例子,{\cf +} 是一个内置加法操作的名字,{\cf 23} 和 {\cf 42} 为操作数。表达式 {\cf (+ 23 42)} 读作 ``23 和
42的和''。复合表达式可嵌套---第二个表达式读作 ``14 和 23 与 42 的积的和''。

如这些例子指出的,Scheme的复合表达式都用前缀记法。结果,需要括号指示结构。于是, 在数学和其它编程语言中常见的``省略'' 括号在Scheme是不容许的。

和很多其它语言一样,空白(包括行结束符)在分隔子表达式时不重要,可用于显示结构。

\chapter{变量和绑定}

Scheme容许标识符表示包含值的位置。这些标识符称为变量。在很多情况,特别是位置在创建后没有修改过时,把变量直接想成其值是有用的。

\begin{scheme}
(let ((x 23)
      (y 42))
  (+ x y)) \ev 65%
\end{scheme}

在本例,{\cf let} 开始的表达式为一个绑定构造。 {\cf let}表达式把{\cf x} 绑定到 23,把{\cf y} 绑定到 42。这些绑定在且仅在{\cf let}的\textit{体}可见。

\chapter{定义}

用 {\cf let} 表达式绑定的变量是\textit{本地}的,因为绑定仅在{\cf let}的体可见。Scheme也容许创建顶层绑定如下:

\begin{scheme}
(define x 23)
(define y 42)
(+ x y) \ev 65%
\end{scheme}

(它们实际上在顶层程序或库``项层''。)

前两个括住的结构为\textit{定义}; 它们创建顶层绑定,绑定{\cf x} 到 23 和 {\cf y} 到 42。定义不是表达式,不能在所有容许表达式的地方出现。并且定义没有值。

绑定服从程序的词法结构:当多个绑定同名,变量指向最近的绑定,由内而外:

\begin{scheme}
(define x 23)
(define y 42)
(let ((y 43))
  (+ x y)) \ev 66

(let ((y 43))
  (let ((y 44))
    (+ x y))) \ev 67%
\end{scheme}

\chapter{过程}

定义也可以用于定义过程:

\begin{scheme}
(define (f x)
  (+ x 42))

(f 23) \ev 65%
\end{scheme}

简单地说,过程是表达式抽象成的对象。在例子中,第一个定义定义了一个叫{\cf f}的过程。 (包围{\cf f x}的括号表示这是过程定义。) 表达式 {\cf (f
  23)} 为一个过程调用,大致``求值 {\cf (+ x 42)} (过程体) ,其中{\cf x} 绑定到 23''.

过程为对象,可传给其它过程:
%
\begin{scheme}
(define (f x)
  (+ x 42))

(define (g p x)
  (p x))

(g f 23) \ev 65%
\end{scheme}

在例子中,{\cf g}的体中 {\cf p}绑定到{\cf f} 而 {\cf x} 绑定到 23来求值,相当于{\cf (f 23)},求值为 65.

事实上,Scheme许多内置操作通过值为过程的变量而非语法提供。例如{\cf +}操作在很多其它语言作特殊语法提供,在Scheme中只是绑定到一个求和过程的常规标识符。{\cf *}等亦然:

\begin{scheme}
(define (h op x y)
  (op x y))

(h + 23 42) \ev 65
(h * 23 42) \ev 966%
\end{scheme}

过程定义并非创建过程的惟一方法。{\cf lambda} 表达式创建一个新的过程对象,不用给名字:

\begin{scheme}
((lambda (x) (+ x 42)) 23) \ev 65%
\end{scheme}

整个表达式为函数调用;{\cf
  (lambda (x) (+ x 42))}求值为一个单参过程,把42加到上面。

\chapter{过程调用和语法关键字}

{\cf (+ 23 42)}, {\cf (f 23)}和 {\cf ((lambda (x) (+ x 42))  23)} 都是过程调用的例子,{\cf lambda} 和 {\cf let} 表达式不是。因为 {\cf let}虽是标识符符但不是变量,而是\textit{语法关键字}。首个子表达式为语法关键字的列表式服从该关键字指定的特殊规则。定义中的{\cf define}标识符也是语法关键字。 所以,定义也不是过程调用。

{\cf lambda} 关键字指定首个子列表为一个参数列表,余下子列表为过程体。在 {\cf let} 表达式,首个子列表为绑定规范,余下子列表构成表达式体。

过程调用和这些\textit{表达式类型}区别在于列表的首个位置:它不是语法关键字的话表达式就是过程调用。Scheme的语法关键字集较少,这使此工作容易。但可以新建语法关键字绑定。

\chapter{赋值}

由定义、{\cf let} 或 {\cf lambda}表达式绑定的Scheme变量不是直接绑定到指定对象,而是绑定到含该对象的位置。这些位置的内容以后可用\textit{赋值}破坏性地修改:
%
\begin{scheme}
(let ((x 23))
  (set! x 42)
  x) \ev 42%
\end{scheme}

在此情况,{\cf let} 表达式体有两个表达式,它们顺序求值,最后一个表达式的值为整个 {\cf let}表达式的值。{\cf (set! x 42)} 是一个赋值,它说``把{\cf x} 指向对象42的位置''。从而上一表达式中{\cf x}值从 23 被改成 42.

\chapter{派生语法和宏}

很多\rsevenrs{}小语言的表达式可重写为更基本的表达式类型。例如{\cf let} 表达式可改写为一个过程调用和一个{\cf lambda}表达式。以下两个表达式等价:
%
\begin{scheme}
(let ((x 23)
      (y 42))
  (+ x y)) \ev 65

((lambda (x y) (+ x y)) 23 42) \lev 65%
\end{scheme}

语法表达式如{\cf let}表达式称为\textit{派生的},因为它们的语义可以通过语法变换从其它表达式得到。一些过程定义也是派生表达式。以下两个定义等价:

\begin{scheme}
(define (f x)
  (+ x 42))

(define f
  (lambda (x)
    (+ x 42)))%
\end{scheme}

Scheme程序可以绑定语法关键字到宏来创建自己的派生表达式:

\begin{scheme}
(define-syntax def
  (syntax-rules ()
    ((def f (p ...) body)
     (define (f p ...)
       body))))

(def f (x)
  (+ x 42))%
\end{scheme}

{\cf define-syntax} 构造指出括住的结构匹配{\cf (def f (p ...) body)},其中{\cf f}, {\cf p}和 {\cf body} 为模板变量,会被转换为{\cf (define (f p ...) body)}。从而,例子中{\cf def} 表达式被重写为:

\begin{scheme}
(define (f x)
  (+ x 42))%
\end{scheme}

新建语法关键字的能力使Scheme极富灵活性和表达力,使其它语言中很多特性可直接在Scheme中实现:任何Scheme程序员可新增表达式类型。

\chapter{语法数据和数据值}

\textit{数据值} 为Scheme对象的子集。包括布尔值、整数、字符、符号、字符串、列表、向量、位向量均为数据值的元素。数据值可以用文本表示为\textit{语法数据},可读写而不丢信息。一般地多个语法数据对应一个数据值。进一步,每个数据值在程序中可平凡地写成字面值表达式,只用在前面加{\cf\singlequote} :

\begin{scheme}
'23 \ev 23
'\schtrue{} \ev \schtrue{}
'foo \ev foo
'(1 2 3) \ev (1 2 3)
'\#(1 2 3) \ev \#(1 2 3)%
\end{scheme}

上例中{\cf\singlequote} 对于符号和列表外字面常量不需要。语法值{\cf foo}表示名为``foo''的符号,{\cf 'foo}为以该符号为值的字面表达式。语法值{\cf (1 2 3)}表示一个有元素1、2、3的列表,{\cf '(1 2 3)} 为以该列表为值的字面表达式。类似地, 语法值{\cf \#(1 2 3)}表示一个有元素1、2、3的列表{\cf \#(1 2 3)},而{\cf '\#(1 2 3)}为对应字面表达式。

语法数据为 Scheme表达式的超集。故数据可用于把Scheme表达式表为数据对象。特别是,符号可用于表示标识符。

\begin{scheme}
'(+ 23 42) \ev (+ 23 42)
'(define (f x) (+ x 42)) \lev (define (f x) (+ x 42))%
\end{scheme}

这方便写操作Scheme代码的程序,特别是解释器和变换器。

\chapter{继续}

Scheme表达式求值时有一个\textit{继续}想要其值。继续表示整个(默认)的未来计算。例如,非正式地 {\cf 3}的继续为表达式
%
\begin{scheme}
(+ 1 3)%
\end{scheme}
%
。正常情况下这些继续被隐藏,程序员不怎么管它们。在少数情况下,程序员需要显式与继续打交道。{\cf call-with-current-continuation}过程让程序员创建重置当前继续的过程。{\cf call-with-current-continuation}过程接受一个过程,立即以\textit{退出过程}为参数调用它。退出过程被调用时参数就成为{\cf call-with-current-continuation}调用的值。即退出过程放弃自己的继续,重置为{\cf call-with-current-continuation}调用的继续。

下例中,表示加1的继续的退出过程绑定{\cf escape},然后用3调用它。{\cf escape}调用的退出过程被放弃,于是3被传给加1的继续:
%
\begin{scheme}
(+ 1 (call-with-current-continuation
       (lambda (escape)
         (+ 2 (escape 3))))) \lev 4%
\end{scheme}
%
退出过程有无限的生存期:它可以在继续后调用且可多次调用。这使{\cf call-with-current-continuation}明显比其它语言典型的非本地控制构造如异常强大。

\chapter{库}

Scheme 代码可组织为称为\textit{libraries}的组件。每个库包含定义和表达式。它可从其它库导入定义和导出定义到其它库。

以下叫 {\cf (hello)} 的库导出{\cf hello-world}定义和导入base库和display库。{\cf hello-world}是在输出一行{\cf Hello World}的过程:
%
\begin{scheme}
(define-library (hello)
  (export hello-world)
  (import (scheme base)
          (scheme display))
  (begin
    (define (hello-world)
      (display "Hello World")
      (newline))))%
\end{scheme}

\chapter{程序}

库由其它库,最终 Scheme \textit{程序}调用。类似于库,程序包含导入、定义和表达式,还有特定的运行起点。从而一个程序通过它导入的库的传递闭包定义一个Scheme程序。

以下程序通过process-context库的{\cf command-line}过程取得首个命令行参数。然后它用{\cf with-input-from-file}打开文件并使之成为当前输入端口,最后自动关闭。然后,它调用{\cf read-line}过程读入一行文本,再用{\cf write-string}和{\cf newline}输出该行,接着循环至文件结束:
%
\begin{scheme}
(import (scheme base)
        (scheme file)
        (scheme process-context))
(with-input-from-file
  (cadr (command-line))
  (lambda ()
    (let loop ((line (read-line)))
      (unless (eof-object? line)
        (write-string line)
        (newline)
        (loop (read-line))))))%
\end{scheme}

\chapter{REPL}

实现可以提供称为交互会话\defining{REPL}(Read-Eval-Print Loop), 其中可一个一个地输入和处理导入、表达式和定义。REPL 开始时导入base库和其它可能的库。实现可提供REPL从文件读输入的操作模式,这文件一般不是程序,因为可以在非起始处导入。

以下是一个短的REPL会话.  {\cf >} 为输入提示符:

\begin{scheme}
> ; A few simple things
> (+ 2 2)
4
> (sin 4)
Undefined variable: sin
> (import (scheme inexact))
> (sin 4)
-0.756802495307928
> (define sine sin)
> (sine 4)
-0.756802495307928
> ; Guy Steele's three-part test
> ; True is true ...
> \#t
\#t
> ; 100!/99! = 100 ...
> (define (fact n)
    (if (= n 0) 1 (* n (fact (- n 1)))))
> (/ (fact 100) (fact 99))
100
> ; If it returns the *right* complex number,
> ; so much the better ...
> (define (atanh x)
    (/ (- (log (+ 1 x))
          (log (- 1 x)))
       2))
> (atanh -2)
-0.549306144334055+1.5707963267949i%
\end{scheme}

%%% Local Variables: 
%%% mode: latex
%%% End: 
